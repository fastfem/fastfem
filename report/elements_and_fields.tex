In order to use the weak formulation, \texttt{FastFEM} needs to know how to integrate functions on a given mesh. Each element type specifies both the shape functions and the algorithm used to integrate a function on that element. For example, the mass matrix entries (\ref{eqn:mass_matrix}) on two different element types are:

\begin{itemize}
\item For linear triangular elements with shape functions $\{\phi_1,\phi_2,\phi_3\}$ having the Kronecker delta property for each vertex ($\phi_i(v_j) = \delta_{ij}$ for vertices $v_1,v_2,v_3$), the analytical solution to the integral is known, since the shape functions are linear.

\item Spectral elements (cit) have shape functions $\phi_{i_1,i_2}(F(x,y)) = L_{i_1}(x)L_{i_2}(y)$ for Lagrange interpolation polynomials $L_i$ based on GLL (Gauss-Lobatto-Legendre) quadrature, where $F$ denotes the coordinate transform from a reference space $[-1,1]^2$ to the element in the mesh's space. Integration is done using GLL quadrature in order to produce a diagonal mass matrix, so the integral is computed
\begin{equation}
\begin{aligned}
    \int_{\Omega_{SE}} \phi_{i_1,i_2}\phi_{j_1,j_2} ~dV = \int_{\Omega_{SE}} L_{i_1}(x)L_{i_2}(y)L_{j_1}(x)L_{j_2}(y) ~dV \\
    \approx \sum_{k_1,k_2=0}^N \alpha_{k_1}\alpha_{k_2} J(F(t_{k_1},t_{k_2}))L_{i_1}(t_{k_1})L_{i_2}(t_{k_2})L_{j_1}(t_{k_1})L_{j_2}(t_{k_2}) \\
    = \alpha_{i_1}\alpha_{i_2}J(F(t_{k_1},t_{k_2}))\delta_{i_1,j_1}\delta_{i_2,j_2}
\end{aligned},
\label{eqn:spectral_mass_matrix}
\end{equation}
where $J = \rho|\det DF|$ is the Jacobian with some scaling parameter $\rho$, which may vary in space. This is relevant for PDEs with a varying mass term.
\end{itemize}


\subsubsection{Elements} \label{sec:elem_field:elements}
To employ the different behaviors, elements inherit from a superclass that with abstract methods for each integral. For $H^1$ elements in 2D, these are


\begin{verbatim}
element.integrate_field(position_field, field, jacobian_scale)
\end{verbatim}

\begin{equation}
\int_{\Omega_E} f~dV
\label{eqn:element_integrate_field}
\end{equation}

\begin{verbatim}
element.integrate_basis_times_field(position_field, field, indices, jacobian_scale)
\end{verbatim}

\begin{equation}
\int_{\Omega_E} \phi_i f~dV
\label{eqn:element_integrate_basis_times_field}
\end{equation}

\begin{verbatim}
element.integrate_grad_basis_dot_field(position_field, field, indices, jacobian_scale)
\end{verbatim}

\begin{equation}
\int_{\Omega_E} (\nabla \phi_i)\cdot \mathbf{f}~dV
\label{eqn:element_integrate_grad_basis_dot_field}
\end{equation}

\begin{verbatim}
element.integrate_grad_basis_dot_grad_field(position_field, field, indices, jacobian_scale)
\end{verbatim}

\begin{equation}
\int_{\Omega_E} (\nabla \phi_i)\cdot (\nabla f)~dV
\label{eqn:element_integrate_grad_basis_dot_grad_field}
\end{equation}

Indices specify which basis functions should be computed. This is helpful for spectral elements, where only diagonal basis entries are needed. These functions implement integration against a field instead of two shape functions. This is because it may be computationally efficient to compute the contracted values $(I_{ij}u^j)_i$ instead of computing the matrix first, then contracting. For example, the stiffness matrix entries (\ref{eqn:stiffness_matrix}) are never used directly for the heat equation (\ref{pde}), but instead are used in contraction with the heat field $f$. This contraction is to produce the integral (\ref{eqn:element_integrate_grad_basis_dot_grad_field}), which can just be called by itself. If uncontracted matrix entries are desired, one can easily integrate each basis function individually.

\subsubsection{Fields} \label{sec:elem_field:fields}

The fields passed into the integrate functions need to have an array structure, with indices for both the basis function index and tensor index. For example, the deformation gradient $DF$ is represented in memory as coefficients $\left(\partial_k F^j\right)_i$, where
\begin{equation}
{(DF)^j}_{k} = \sum_i\left(\partial_k F^j\right)_i\phi_i,~~~F=F^j\mathbf{e}_j
\label{eqn:def_grad_discrete}
\end{equation}
when it is in the same space as the shape functions. Some elements may function better with multi-indexed shape functions (such as spectral elements), which further complicate the coefficient indices. Here, the field is represented with two sets of indices: the tensor indices and the shape indices. This combined with the \emph{field stacking} feature we discuss below to vectorize function calls, introduces the need to keep track of which index corresponds to what.
To do this, we utilize a \verb+Field+ wrapper class, which takes a \emph{numpy} or \emph{JAX} array with the corresponding tensor, shape function, and stack index shapes. This has the added benefit of allowing the delegation of operations to \emph{JAX} or \emph{numpy} depending on if the \emph{JAX} feature set is necessary (say, for \emph{autodiff}).

\paragraph{The Field Object}

Each field has three shapes corresponding to it:
\begin{itemize}
\item \verb+point_shape+ - The shape of the (pointwise) tensor the field represents.
\item \verb+basis_shape+ - The shape of the basis indices used to reference the shape functions.
\item \verb+stack_shape+ - The shape of the \emph{field stack}, allowing for vectorized operations.
\end{itemize}

In addition, we store whether or not to use \emph{JAX} functions. If \verb+use_jax+ is true for a field, then functions that delegate to \verb+numpy+ call \verb+jax.numpy+ instead. Additionally, any field resulting from an operation will inherit \verb+use_jax+ from any argument that has \verb+use_jax == True+.

\paragraph{Field Stacking}

Due to the overhead of Python, vectorizing operations is important for performance. \texttt{FastFEM} does this by \emph{field stacking}, so that a single operation can be performed on multiple fields at the same time. This is important for integrating on meshes (\ref{sec:elem_field:mesh_element}), where multiple integration calls must be made on one type of element.

\subsubsection{The Mesh Element} \label{sec:elem_field:mesh_element}

In addition to triangular and spectral elements, we have triangular mesh elements, which take a mesh from the mesher (\ref{sec:mesher}) and generate an element. The corresponding basis is one dimensional, where each $\phi_i$ corresponds to a node (vertex) of the mesh. Each sub-element $\Omega_k$ corresponds to a polygon of the mesh. With vertices $i_1,i_2,i_3$, we relate the basis on the sub-element $\{\tilde\phi_1,\tilde\phi_2,\tilde\phi_3\}$ to the basis on the mesh element by:

\begin{equation}
    \left.\phi_{i_j}\right|_{\Omega_k} = \tilde\phi_j,~~~~j=1,2,3
    \label{eqn:mesh_subelement_assembly_basis}
\end{equation}

Integrating a field on the mesh element is done as follows:
\begin{enumerate}
\item Create a new field with the sub-element's basis shape and an axis of size $N$ prepended to the stack shape, where $N$ is the number of sub-elements in the mesh. Values are populated according to the assembly rule (\ref{eqn:mesh_subelement_assembly_basis}).
\item Call the corresponding sub-element's integration function. Let $I_k$ denote the result on sub-element $k$.
\item Perform an atomic addition. In the case of \verb+element.integrate_field+, this is just an addition reduction. For basis integrals $I_k(\tilde\phi_j)$, the coefficient to $\phi_{i_j}$ is accumulated.
\end{enumerate}