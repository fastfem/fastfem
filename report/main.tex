\documentclass[headings=standardclasses, abstract=true]{scrartcl}

\usepackage{graphicx} % Required for inserting images
\usepackage[sorting=none, style=ieee]{biblatex} % Required for bibliography
\usepackage{amsmath} % for math equations
\addbibresource{../docs/assets/bibliography.bib} % Path to the bibliography file
\usepackage[
    pdfauthor={Sina Atalay},
    pdftitle={RenderCV: A LaTeX CV/Resume App for Academics and Engineers},
    hidelinks=true
]{hyperref} % Required for hyperlinks and metadata

\title{FastFEM}
\subtitle{A Python package for solving PDEs with the finite element method}
\author{
    Sina Atalay\textsuperscript{1*}, Sacha Escudier\textsuperscript{1*}, Kentaro Hanson\textsuperscript{1*} \\
    {\footnotesize \textsuperscript{1}Princeton University, Princeton, NJ, USA}\\
    {\footnotesize \textsuperscript{*}All authors contributed equally}
}
\date{
    \normalsize December 2024
}


\begin{document}

\maketitle

\begin{abstract}
\noindent Lorem ipsum dolor sit amet, consectetur adipiscing elit. Vestibulum congue gravida sem non dictum. Aenean sit amet mi fermentum ante laoreet dictum sit amet a magna. Praesent sed aliquet dui. Vivamus scelerisque condimentum mauris id euismod. Duis elementum urna eu rutrum mattis. Donec fermentum, risus et viverra aliquet, ante sapien consequat augue, sit amet dictum est nisl id elit. Proin dapibus congue tincidunt. Pellentesque habitant morbi tristique senectus et netus et malesuada fames ac turpis egestas. Fusce felis eros, aliquam sed dui ut, tempor condimentum neque. Donec quis sapien bibendum, faucibus urna sed, gravida felis. Praesent et quam ligula. Aliquam ac est eu odio tincidunt volutpat a in augue. Maecenas fermentum velit felis, vel viverra dui scelerisque pharetra. Nullam eros lorem, finibus pulvinar eleifend eu, pulvinar ac ipsum. Sed velit neque, venenatis sit amet urna vitae, vulputate ullamcorper est.
\end{abstract}

\section{Introduction}

Partial differential equations (PDEs) are the fundamental tools for mathematically modeling natural phenomena. Many fundamental phenomena observed in nature, such as general relativity \cite{Marolf2001}, quantum mechanics \cite{Feit1982}, heat diffusion \cite{Bergman2011}, fluid mechanics \cite{Lukaszewicz2016}, pricing of financial derivative contracts \cite{Barles1998}, structural analysis \cite{Boresi2002}, and electromagnetism \cite{Griffiths2017}, are described by PDEs. However, most of these PDEs do not have closed-form solutions, especially in complex geometries. Therefore, engineers have developed many numerical methods for solving PDEs. One of the most popular numerical methods among them is the finite-element method (FEM), which originated in the early 1940s \cite{Liu2022}. FEM is capable of solving non-linear PDEs in highly complex geometries. Since the 1940s, FEM has undergone significant advancements and has revolutionized the way scientific modeling and engineering design. Today, it is widely used in many industrial applications.

Currently, there are many open-source FEM software out there \cite{fem_getdp, fem_agros, fem_calculix, fem_elmerfem, fem_freefem, fem_goma, fem_fenicsx, fem_dealii}. With \texttt{FastFEM}, we attempt to develop another FEM with the goal of

\begin{itemize}
    \item Creating an easy-to-use and clean Python interface
    \item Using modern tools like JAX \cite{jax2018github} for advanced array computing with automatic differentiation capabilities
\end{itemize}

One of the motivations for creating a Python interface was the capability of the language to create very intuitive-to-use interfaces. A modern Python interface can offer users a great way of describing FEM problems. The other motivation was leveraging the existing scientific Python environment. Python's popularity in the scientific world is still increasing, and modern libraries with state-of-the-art technologies like JAX, PyVista \cite{Sullivan2019}, etc., are being developed.


FastFEM is planned to be a big project, but as the goal of \texttt{v0.0.1}, we decided to focus on 2D parabolic PDEs. Parabolic PDEs, such as the heat diffusion equation, Poisson's equation, and the Black-Scholes equation, are very common in various applications. A 2-dimensional parabolic PDE can be expressed as

\begin{equation*}
    \frac{\partial^2 f(x,y,t)}{\partial x^2} + \frac{\partial^2 f(x,y,t)}{\partial y^2}
    =
    h(f) \frac{\partial f(x,y,t)}{\partial t} + g(x,y)
    \label{pde}
\end{equation*}
where $f(x,y,t)$, $h(f)$, and $g(x,y)$ are scalar functions, $x$ and $y$ are spatial coordinates, and $t$ is time.

\texttt{FastFEM v.0.0.1}
\begin{enumerate}
    \item creates a 2D mesh,
    \item takes initial conditions, boundary conditions, h(f), and g(x,y) as inputs,
    \item solves the PDE with FEM, and
    \item plots the solution.
\end{enumerate}


In this report, the theory behind FEM is summarized. The features and capabilities of FastFEM are presented. Some example results are shown. Finally, the conclusion and outlook are discussed.

\section{Theory}

\section{Features and Capabilities}

\subsection{Mesher}

\subsection{Elements and Fields}

\subsection{Plotter}

\section{Results}

\section{Conclusion}

\printbibliography

\end{document}